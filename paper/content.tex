% status: 0
% chapter: TBD

\title{Oracle NoSQL}


\author{Lin, Qingyun}
\affiliation{%
  \institution{Indiana University Bloomington}
  \city{Bloomington} 
  \state{Indiana} 
  \postcode{47408}
  \country{USA}
}
\email{ql10@iu.edu}

\author{Gregor von Laszewski}
\affiliation{%
  \institution{Indiana University}
  \streetaddress{Smith Research Center}
  \city{Bloomington} 
  \state{IN} 
  \postcode{47408}
  \country{USA}}
\email{laszewski@gmail.com}


% The default list of authors is too long for headers}
\renewcommand{\shortauthors}{G. v. Laszewski}


\begin{abstract}
This paper is an Introduction about Oracle NoSQL technology. First, the concept of Oracle NoSQL will be demonstrated. Second, the architecture of Oracle NoSQL technology will be generally introduced. And the application, performance usecases about this technology will be demonstrated as follows.
\end{abstract}

\keywords{Oracle NoSQL, architecture, usecases, application, performance}


\maketitle

\section{Introduction}

Oracle NoSQL provides a scalable, distributed NoSQL database. Highly reliable, flexible and available are three main properties of data management service it provides.

As the cloud computation being developed, the requirement of distributed storage system is also getting more significant. Even though traditional relational database is friendly to the storage space and highly reliable, but it is not convienient enough to be scaled. Based on this situation, the NoSQL database is designed. 
NoSQL is designed as high-performance, non-relational databases. NoSQL utilizes a partition key to retrieve values, column sets, or semi-structured JSON, XML, or other documents containing related item attributes. NoSQL is designed to scale “out” using distributed clusters of low-cost hardware to increase throughput without increasing latency.
Based on the NoSQL concept, Oracle NoSQL Database is designed. Oracle NoSQL Database is a NoSQL-type database from Oracle Corporation. It provides transactional semantics for data manipulation, easy scalability, and simple management.
The definition of NoSQL Database is still unclear now, it is either not only a SQL-based relational database system or a simply not SQL-based RDBMS. But all these definitions explain not enough about what NoSQL is. Compared to the traditional RDBMS, characterized by ACID: Atomicity, Consistency, Isolation, and Durability, the NoSQL can be explained by BASE acronym:
Basically Available, means replication helping to reduce the probability of the data unavailability, and making any remaining failures partial. The system will be always availabe under this property, even though some nodes fail sometimes.
Soft state: Compared to ACID, the NoSQL systems allow data to be inconsistent and relegate.
Eventually consistent: Although soft state does exist in the system, NoSQL ensure that the whole system will be consistent again in the future. The difference between the traditional RDBMS and NoSQL is that NoSQL guarantees consistency only in the unclearly defined future. 


\section{Architecture}

Oracle NoSQL Database is based on the Oracle Berkeley DB Java Edition high-availability storage engine. It adds an addition layer of services for distributed environments to provide a distributed, highly available key/value storage, suited for large-volume, latency-sensitive applications.
In this respect, an Infoworld review mentions that Oracle bought the company which developed the Berkeley DB, Sleepycat Software.\cite{hid-sp18-515-editor00}.
Oracle NoSQL Database is a client-server system. Each of the nodes in the system has the replicated shard data, which comprise the shard. A simple key-value paradigm is provided to the application developer. And the majority part of the key for a record is hashed to identifiy the shard which owns the record. Oracle NoSQL Database is designed to support dynamical numbers of shards, which means additional hardware should be easy to apply. The key-value pairs will be redistributed across the new set of shards without shutdown the system if the number of shards changes. A shard is controled by a master node which controls reading and writing requests of the shard. Streaming replication is used to keep up replicas to date. Every single change on the master node is recorded locally in disk.
Oracle NoSQL Database is based on single-master, multi-replica structure. When the master replica node fails, a consenus will minimizes downtime, which is based on PAXOS. When the failed node is repaired, it will rejoin the shard, and is rolled forward to the date and becomes available again. Therefore, Oracle NoSQL Database server is high available against the failures of nodes. 
Oracle NoSQL Database partitions the data and distributes it to the storage nodes in a balance way. It is routing read and write operations to the most suitable node for optimizing the performance.
Avro data serialization is supported by Oracle NoSQL Database, which means a compact, schema-based binary data format is provided.
Oracle NoSQL Database includes a topology planning feature, means that an administrator can modify the configuration of database without shuting down the database.Increasing data distribution, Relication Factor and Rebalancing Data Store are executable while the database is still online. 
An administration service is provided by Oracle NoSQL Database, which can be accessed from web console or command-line interface. 


\section{Application}

Oracle NoSQL Database suppoerts multiple applications. 
Java, C, Python, REST APIs are supported by Oracle NoSQL Database. These APIs allow developers to manipulate data stored in Oracle NoSQL Database. And they also include Avro support, which means developers can serialize key-value records and de-serialize keyvalue records interchangeably between C and Java applications.\cite{hid-sp18-515-editor01}
Oracle Big Data SQL is a SQL access layer to data stored in Oracle NoSQL database. It can also access the data in Hadoop, HDFS, Hive. User can run query Oracle NoSQL data from Hive or Oracle Database with this layer, and map reduce jobs against data stored in Oracle NoSQL Database can also be run by Oracle Big Data SQL. 
Oracle Corporation adds support for Oracle REST Data Services in the Oracle NoSQL Database 12.1.3.2.5 version, which allows customers to build a REST-based application that can manipulate data stored either in Oracle Database or Oracle NoSQL Database. 
Large Objects such as audio and video files can also be read and written by Oracle NoSQL Database based on the stream APIs. The value in LOBs' entirety have not to be materialized in memory. These APIs decrease the latency of operations across mixed workloads of objects of varying sizes. 
Oracle NoSQL Database also supports KVAvroInputFormat and KVInputFormat to read from Oracel NoSQL Database into Hadoop Map/Reduce jobs. 
External table can be fetched by Oracle NoSQL data from Oracle database using SQL statements such as Select, Select Count(*) etc Developer can manipulate the data via standard JDBC drivers and visualize it through enterprise Business Intelligence tools when the external tables are accessed by Oracle NoSQL Database. 
Oracle Event Processing is used to read Oracle NoSQL Database via the NoSQL Database cartridge. CQL queries can be used to query the data when the cartridge is configured. Large volumes of RDF data can be also accessed by Jena Adapter, which is developed by Oracle Semantic Graph. This adapter accesses to graph data stored in Oracle NoSQL Database in an efficient way via SPARQL queries. 
Facilities are provided by Oracle NoSQL Database for performing a rolling upgrade, which allows all of the nodes in the system to be upgraded in the database cluster without shuting down the database and keep the system availabel to clients. 
Greater protection from unauthorized access to sensitive data is provided from OS-indenpendent, cluster-wide password-based user authentication and Oracle Wallet integration. SSL encryption and network port restrictions is used to improve protection from network. 
The most common version of Oracle NoSQL Database is released updates (4.0), the new features are as follows:
- Full text search - with Elastic Search, Oracle NoSQL Database can perform full text searches.
- Time-To-Live - easily to target and cut off the "expired" data.
- SQL Query - help developer easier to query using SQL than API.
- Import/Export - convenient to backup/resore and transform data among different Oracle NoSQL Database. 

\section{Performance}
This is a experiment about Oracle NoSQL Database and Yahoo! Cloud Serving Benchmark(YCSB), demonstrating how the system scales with the number of nodes in the system. All the results is from this pdf file.\cite{hid-sp18-515-editor02}.
A constant YCSB load per storage node to configurations of varying sizes. Each storage node owned an Intel Xeon Processor X5670 dual socket machine with 6 cores/socket and 24 GB of memory. Each machine ran RedHat 2.6.18-164.11.1.el5.crt1. The disk size is 300GB, which limits the scale size. 100M records was held by each node to dictate the overall configuration, with an average key size of 13 bytes and data size of 1108 bytes. 
The performance of raw insert in Oracle NoSQL Database for configurations ranging from a single replication group system with three nodes storing 100 million records to a system with 32 replication groups on 96 nodes storing 2.1 billion records(the YCSB benchmark is limited to a maximum of 2.1 billion records). The result shows that Throughput of system has almost the linear relationship with the database size and number of replication groups grows, even though the response time increase slightly. The response time for a workload of 50\% reads and 50\% updates was also evaluated. Both the update and read latency decline while the system grows in size(data size and replication groups number), and the throughput scales was still almost linearly satisfying the scalability needed for nowaday's demanding applications. 

\section{Usecases}
The good use cases for Oracle NoSQL Database will share these common properties: stringent latency requirements for simple queries, handling a high velocity of requests, the ability to tolerate eventually consistent data, and the ability to scale the load linearly across a cluster of commodity servers. Based on these concepts, the followings are good cases of Oracle NoSQL Database:
1. Online display advertising - Storing behavioral segments in a NoSQL database is a common use case here.  These tend to be high velocity, low latency requests where the data being utilized is essentially leveraged to increase the probability that a user will click on a display ad.  Therefore eventual consistency is OK in this use case.  Scaling the cluster to scale with the load is absolutely essential in this use case.
2.  Scalable email system - In this use case using the Oracle NoSQL Database shard local ACID transactions is really helpful.  It's convenient to model a key/value solution to address the scale and transactional requirements of this type of workload.  
3.  Online fraud detection - Another great use case for a Oracle NoSQL solution in which a fraud model must be retrieved  scored, and updated with the latest transaction  all within a very small latency window.

\begin{acks}

  The authors would like to thank Dr.~Gregor~von~Laszewski for his
  support and suggestions to write this paper.

\end{acks}

\bibliographystyle{ACM-Reference-Format}
\bibliography{report} 

