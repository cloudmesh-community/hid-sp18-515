% status: 0
% chapter: TBD

\title{Classification of different periods of time based on the Measurements of electric power consumption using REST service and Docker packaged service}


\author{Qingyun Lin}
\affiliation{%
  \institution{Indiana University}
  \streetaddress{Smith Reserch Center}
  \city{Bloomington} 
  \state{IN} 
  \postcode{47408}
}
\email{ql10@iu.edu}


% The default list of authors is too long for headers}
\renewcommand{\shortauthors}{G. v. Laszewski}


\begin{abstract}
This paper provides the proposal of the final project. This proposal contains the goal, design, scenarios and the Technology Stack of the project
\end{abstract}

\keywords{hid-sp18-515, classification, REST, docker swarm, kubernetes}


\maketitle

\section{Introduction}

What is docker swarm and kubernetes? Docker swarm and kubernetes are computer programs that performs operating system level virtualization, also known as containerization. The container image is a lightweight, self-contained executable package that contains everything needed to run it: code, runtime, system tools, system libraries, settings. Regardless of the environment, containerized software can run the same Linux and Windows applications. The container isolates the software from its surroundings, such as the differences between the development environment and the staging environment, and helps reduce conflicts between teams that run different software on the same infrastructure. Therefore, project running in container file can be apply to other environment easly. REST is an architectural style that defines a set of constraints and attributes based on HTTP. A REST-conformant Web service or RESTful Web service provides interoperability between computer systems on the Internet. RESTful web services allow the requesting system to access and manipulate textual representations of web resources by using a uniform and predefined set of stateless operations. Therefore, deploying the project both in docker and REST, helps the project easy to transfer and the result of the project easy to be accessed. And electric consuming problem is a heated issue both for the goverments and consumers. Therefore, get more understanding of the features of electric measurement is useful for the environment. To sum up, this project is meaningful.


\section{Design}

Use REST to display the result, delploy the project in docker swarm and kubernetes containers.
Use python to classify the dataset into groups.

\section{Scenarios}

The scenarios of this project is as followings:

1. Download the necessary dataset household\_power\_consumption.txt\cite{editor00}.

2. Clean the data to be complete to train.

3. Analyze the data, and generate a suitable classification.

4. Generate suitable graph to discribe the classification.

5. Draw conclusion of the project.


\section{Technology Stack}

The technology stack of this project is as followings:

1. Python will be used to analyze and generate graph.

2. The result will be displayed in REST

3. The project will be deployed in docker and kubernetes

4. Certain machine learning algorithm will be applied


\section{Conclusion}

REST helps the result easy to access, the docker swarm and kubernetes help the project easy to be transformed.


\begin{acks}

  The authors would like to thank Dr.~Gregor~von~Laszewski for his
  support and suggestions to write this paper.

\end{acks}


\bibliographystyle{ACM-Reference-Format}
\bibliography{report} 

