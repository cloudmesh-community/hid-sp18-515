
% status: 0
% chapter: TBD

\title{Big Data in Pychology}


\author{Qingyun Lin}
\affiliation{%
  \institution{Indiana University}
  \streetaddress{Smith Reserch Center}
  \city{Bloomington} 
  \state{IN} 
  \postcode{47408}
}
\email{ql10@iu.edu}


% The default list of authors is too long for headers}
\renewcommand{\shortauthors}{G. v. Laszewski}


\begin{abstract}
This is the draft of the final paper in I524. This paper will 
introduce what is pychology and what is big data technique simply, 
and why these two fields can be combined togethor, how can the 
society benefit from this combination, and introduce several 
combination examples and what is the future of this field.
\end{abstract}

\keywords{hid-sp18-515, big data, pychology, combination, future}


\maketitle

\section{Introduction}

Pychology problem is getting more and more attention in the morden 
society. And as the big data technology is such a heated topic today, 
the combination of this two fields should be well applied.

Big data is particularly good at dealing with issues that people may 
be reluctant to answer surveys. Often, the way people interact with 
computers reveals the interest they would not express in interviews 
or anonymous surveys\cite{editor00}.

Therefore, it is easier and even more justified for big data technique 
being applied in pychology fields than the traditional statistic methods.

\begin{acks}

  The authors would like to thank Dr.~Gregor~von~Laszewski for his
  support and suggestions to write this paper.

\end{acks}


\bibliographystyle{ACM-Reference-Format}
\bibliography{report} 

