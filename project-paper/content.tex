% status: 0
% chapter: TBD

\title{Big Data in Pychology}


\author{Qingyun Lin}
\affiliation{%
  \institution{Indiana University}
  \streetaddress{Smith Reserch Center}
  \city{Bloomington} 
  \state{IN} 
  \postcode{47408}
}
\email{ql10@iu.edu}


% The default list of authors is too long for headers}
\renewcommand{\shortauthors}{G. v. Laszewski}


\begin{abstract}
There is much discussion about the value of company big data. 
For example, Amazon will match your shopping and page views with 
other shoppers’ purchases and pageviews, and try to find people with 
similar interests. Then Amazon recommends buying products that people 
like, assuming you will like them.

Can big data be applied to answer the research community’s interest 
in psychology?

This paper will introduce what is pychology and what is big data 
technique simply, and why these two fields can be combined togethor, 
how can the society benefit from this combination, and introduce 
several combination examples and what is the future of this field.
\end{abstract}

\keywords{hid-sp18-515, big data, pychology, combination, future}


\maketitle

\section{Introduction}

Pychology problem is getting more and more attention in the morden 
society. And as the big data technology is such a heated topic today, 
the combination of this two fields should be well applied.

Big data is particularly good at dealing with issues that people may 
be reluctant to answer surveys. Often, the way people interact with 
computers reveals the interest they would not express in interviews 
or anonymous surveys\cite{editor00}.

Significant advances in computing technology, coupled with the 
proliferation of large digital networks and social media platforms, 
have produced an unimaginable amount of information about people. 
It is estimated that the amount of digital data currently present is 
in the thousands of exabytes or 10 to the 18th power of bytes.

This era of big data is likely to change the way psychologists observe 
human behavior. However, just as it creates new opportunities, access 
to a large amount of information also brings new challenges to research. 
Michael N. Jones of Indiana University Bloomington introduced the 
Big Data Project at the 2014 APS Annual Conference.

Each little piece of data is a trace of human behavior and offers 
us a potential clue to understanding basic psychological principles, 
said Jones. But we have to be able to put all those pieces together 
correctly to better understand the basic psychological principles 
that generated them\cite{editor01}.

The study of language development as one of Jones's own research interests is a good example of big data research. Collecting a large number of infant samples from the natural environment is time-consuming and often results in small sample sizes. The test theory of the way children learn languages is a long time.

Big data can help speed up the process. As a proof of concept, Jones showed that based on associative learning theory, more than 100,000 words from natural language can be entered into a computer model - the idea is that children group words together according to how often they use them near other words. Jones said that as the analysis progressed, the model did realize that the relationship between 'computers' and 'data' and word classes was closer than 'computers' and 'obsolete'.

Jones said that in the end, a similar analysis can be used to study the theory of connected learning in direct child conversation samples. As long as they have enough data to continue, these models are very good at learning from noise, he said\cite{editor01}.

Tanzeem Choudhury, an information scientist at Cornell University, 
said that big data may help researchers reach the point where they 
can collect behavioral information without sampling human participants. 
Technologies such as smart phones and wearable sensors can collect 
information about physical activity, social interaction, geographic 
location, and more.

The result of this data collection is that it is effective for users; 
it does not require their time or effort, and it greatly reduces 
self-reporting errors.

We can continuously get measurements of behavior without bugging 
people to fill out surveys, said Choudhury. We can potentially get 
continuous measurement without actually having to engage users all 
the time and rely on their self-input\cite{editor01}.

Chowdhury participated in some of these projects. StressSense can track daytime stress and help them avoid anxiety. For example, MyBehavior uses physical activity patterns to suggest ways of maintaining the shape - for example, working more frequently along lines that the user seems to like. MoodRhythm allows patients with bipolar disorder to monitor sleep and social interactions to maintain emotional and energy balance, which is a major improvement in pen and daily behavioral tracking. (These programs are still under development because of smartphone apps.)

Big data has produced positive changes in the online search field. Susan T. Dumais of Microsoft Research Inc. said that the billions of Internet searches that happen each day leave the behavior log that analysts use to improve search engines. Without such a large record, sites such as Google and Bing will never be able to use 2.4 words in an average Internet search and turn it into useful content.

Behavioral logs allow us to characterize, with a richness and fidelity that we’ve never had before, what it is people are trying to do with the tools and systems they’re interacting with, said Dumais\cite{editor01}.

By mining behavior logs, analysts can create personalized algorithms to improve the user's search experience. For example, if Dumais searches for 'sigir', she may need the homepage of the Information Search Special Interest Group (SIGIR). If Little Stuart Bowen performs the same search, he may need the site to hold his position: Special Inspector General for Reconstruction of Iraq (also known as SIGIR).

In other words, the system can know that words and acronyms do not always predict the best way users want to search. Modeling the search in a way that considers the context of the published query is very important for improving Web search. Previous search activity is important, and the location and time of the query are also important. For example, the search for 'Spring US Open' may refer to a golf ball, while the same search in late summer may refer to tennis.

Before you were able to collect Big Data, the person who spoke loudest, or the highest-paid person’s opinion, would dominate, said Dumais. Now the data, especially when derived from carefully controlled Web-scale experiments, dominates.\cite{editor01}

Big Data also allows researchers to rethink past issues in new ways, 
said Brian M. D'Onofrio, an APS researcher at Indiana University Bloomington. 
In particular, he pointed out that researchers should consider re-adjusting 
data that may be collected for other reasons. Reusing big data samples can 
help researchers generate insights. Traditional samples can't achieve the 
statistical power that many labs lack. This is a big challenge for psychology 
to improve its methods and replication process.

With Big Data, it gives you the opportunity to use several different 
types of quasi-experimental designs, to help rule out alternative 
explanations, D’Onofrio said\cite{editor01}.

D'Onofrio and his collaborators recently rearranged the millions of personal records compiled by Sweden in order to challenge the traditional notion that smoking during pregnancy directly leads to subsequent negative behavioral outcomes such as crime. In one study, the researchers analyzed 50,000 siblings whose mothers smoked but did not smoke during pregnancy. They determined the family context factor - not smoking during pregnancy - related to criminal convictions. This recognition can greatly improve interventions: In this case, allowing women to quit smoking should only be part of broader social services.

Tal Yarkoni of the University of Texas at Austin says big data can 
even help psychologists study research. Yarkoni and others recently 
developed Neurosynth, an online program for analyzing a large number 
of fMRI data to guide users to topics of interest. Yarkoni said that 
so far, Neurosynth has synthesized studies from more than 9,000 
neuroimaging studies and about 300,000 brain activations.

One of Neurosynth's main goals is to distinguish brain activity that 
is always associated with a specific psychological process but not 
specific, and high-probability brain activity that means having a 
specific mental process. For example, painful physical stimulation 
may continue to produce patterns of brain activity, but this pattern 
of activity does not necessarily mean the presence of pain; other 
mental states may produce similar patterns. Infer mental processes 
from observed brain activity, this process is called 'reverse 
reasoning', it is difficult to do in a single neuroimaging study.

Neurosynth overturned reasoning by accumulating a large number of 
images and research data in one place. For example, even if some 
active brain regions overlap in these three cases, the database can 
help researchers find areas of the brain that are particularly relevant 
to pain, not working memory or emotions. Tests have shown that, in many 
cases, Neurosynth's performance and analysis are done manually by 
screening research literature - saving hundreds of hours of research 
time compared to just pressing a button, Yarkoni said.

That’s the long-term goal, Yarkoni said. To do this in a quantitative, 
automated way instead of a manual, qualitative way\cite{editor01}.

The long term goal is the theme of the plan, because big data is 
still the existence of science. Not everyone believes that it will 
bring a paradigm shift. Even if big data does change statistical analysis, 
it cannot replace Strong behavioral theory or experiment. But in terms of 
improving these theories or strengthening big data for these experiments, 
researchers cannot ignore it.

\begin{acks}

  The authors would like to thank Dr.~Gregor~von~Laszewski for his
  support and suggestions to write this paper.

\end{acks}


\bibliographystyle{ACM-Reference-Format}
\bibliography{report} 

